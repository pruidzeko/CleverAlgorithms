% The Clever Algorithms Project: http://www.CleverAlgorithms.com
% (c) Copyright 2010 Jason Brownlee. Some Rights Reserved.
% This work is licensed under a Creative Commons Attribution-Noncommercial-Share Alike 2.5 Australia License.

% This is a chapter

\renewcommand{\bibsection}{\subsection{\bibname}}
\begin{bibunit}

\chapter{Physical Algorithms}
\label{ch:physical}
\index{Physical Algorithms}

\section{Overview}
This chapter describes Physical Algorithms.


% biological
\subsection{Physical Properties}
Physical algorithms are those algorithms inspired by a physical process. The described physical algorithm generally belong to the fields of Metaheustics and Computational Intelligence, although do not fit neatly into the existing categories of the biological inspired techniques (such as Swarm, Immune, Neural, and Evolution). In this vein, they could just as easily be referred to as nature inspired algorithms.

The inspiring physical systems range from metallurgy, music, the interplay between culture and evolution, and complex dynamic systems such as avalanches. They are generally stochastic optimization algorithms with a mixtures of local (neighborhood-based) and global search techniques.

%
% Extensions
%
\subsection{Extensions}
There are many other algorithms and classes of algorithm that were not described inspired by natural systems, not limited to:

\begin{itemize}
	\item \textbf{More Annealing}: Extensions to the classical Simulated Annealing algorithm, such as Adaptive Simulated Annealing (formally Very Fast Simulated Re-annealing) \cite{Ingber1989, Ingber1996}, and Quantum Annealing \cite{Apolloni1989, Das2005}.
	\item \textbf{Stochastic tunneling}: based on the physical idea of a particle tunneling through structures \cite{Wenzel1999}.
\end{itemize}

\putbib
\end{bibunit}

\newpage\begin{bibunit}\input{a_physical/simulated_annealing}\putbib\end{bibunit}
\newpage\begin{bibunit}\input{a_physical/extremal_optimization}\putbib\end{bibunit}
\newpage\begin{bibunit}\input{a_physical/harmony_search}\putbib\end{bibunit}
\newpage\begin{bibunit}\input{a_physical/cultural_algorithm}\putbib\end{bibunit}
\newpage\begin{bibunit}\input{a_physical/memetic_algorithm}\putbib\end{bibunit}



